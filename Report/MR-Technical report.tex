\documentclass[10pt,twocolumn,letterpaper]{article}

%% Language and font encodings
\usepackage[english]{babel}
\usepackage[utf8x]{inputenc}
\usepackage[T1]{fontenc}

% xcolor is the package between braces and between brackets is the option that we want to use
\usepackage{fancyhdr}
\usepackage{biblatex}
\usepackage[color]{showkeys}
\usepackage{etoolbox}
\usepackage{graphicx,tabularx}
\usepackage{amsfonts} 

%% Sets page size and margins
\usepackage[a4paper,top=3cm,bottom=2cm,left=3cm,right=3cm,marginparwidth=1.75cm]{geometry}

%% Useful packages
\usepackage{amsmath}
\usepackage[colorlinks=true, allcolors=blue]{hyperref}

\fancypagestyle{plain}{%
  \fancyhead{}
  \fancyfoot{}
  \fancyfoot[R]{\thepage}
  \fancyfoot[L]{[INFOMR] Multimedia Retrieval - Utrecht University}
}
\pagestyle{plain}

\title{%
  3D Mesh Retrieval System \\
  \large Multimedia Retrieval \\
    Utrecht University}
    
\author{
  Fabien Da Costa Barros [0720823]\\
  \url{f.n.dacostabarros@uu.students.nl}
  \and
  LastName2, FirstName2\\
  \url{first2.last2@xxxxx.com}
}

\date{\today}
\addbibresource{refs.bib}                                                                     
\begin{document}

\maketitle

\selectlanguage{english}

\section*{Introduction}

The objective of this project is to choose, parametrise and put in practise of different procedure and algorithms to build a end-to-end MR System. More precisely the focus will be put on 3D mesh retrieval system. From one mesh we have to retrieve all (at least a decent proportion) of similar mesh. The main challenges will be to analyse different features of a 3D mesh. The main steps of the assignments will be to read and display a 3D mesh; then we have to fix and normalise all the 3D meshes in our database. After that we will extract the feature of the meshes to get the essence of what can define the object is in real life to compare this and retrieve similar mesh. At the end we have to improve it so that it can work faster. All of the code can be find on this \href{https://github.com/FDaCostaB/3DMeshRetrievalSystem}{GitHub page}  

\section{Selection and setup the work environment}

In this section, four different related packages have been tested and compared. The Comparison of these packages has been done based on several features that might be useful in the future sections of this project. For example, being able to re-mesh objects, scaling them, rotating them, and moving them are the feature that can be useful in the process of normalizing the 3d objects in the database. Also, having some information about the 3d objects, like the number of vertices and faces, is necessary for understanding if these 3d objects are normalized or not. On the other hand, this package must support different formats that are usually used to store 3d objects.
In this case, we narrowed down the packages that work with Python language and well-known between the community. So, we tested and analyzed "Pymesh", "PyMeshLab", "TriMesh", and "Open3D". Here are the results:\\ \\
	We wanted to be able to open "PLY", "OFF", "OBJ", and "STL" formats. "PLY" and "OFF" formats were discussed in the lecture, while "OBJ" and "STL" are really common. All of them were able to open "PLY", "OFF", "OBJ", and "STL" formats, so that was not really discriminating. \\ \\
	We tried to compare the different re-meshing, repairing and extraction features with our current knowledge and thought that PyMeshLab and Open3D seems to be quite good. The complexity was also one of our requirements as from previous experience less complex library offers less options. \\ \\
	After achieving step 1 with four of those libraries, we try to find a balance between number of dependencies, options proposed for visualizing and analysis. Then we decide to choose "PyMeshLab"\cite{pymeshlab} \\ \\
	
\section{Step 1}

	For this step, we have done several tries with PyMesh, PyMeshLab, TriMesh and Open3D to get in touch with the different tools. We get some trouble to set up PyMesh as we have no previous knowledge on docker. That was one of the main issue that we encounter, thanks to this we are thinking to dockerise our application for the assignment so that it can be easier to run on different OS and avoid installing all the dependencies to run it. For the other libraries it was really easy to set up and we mainly wopy-paste the code snippet of the 'Getting started' page of the corresponding library.  Later on  we will only discuss the PymeshLab project. \\ \\
	Our project have three main dependencies : PyMeshLab, Polyscope and Numpy and run with Python 3.9. At the moment, we have two scripts : main.py and render.py. To have a nice overview and organisation of  the project we will try to split as much as possible the different functionalities (mainly different steps) of the project. 'main.py' just verify the number of arguments and call the function render from 'render.py' with the first arguments passed through. This arguments is the name of the file. We prefer to pass it through the command line so that's it make easy to check tens of mesh and quickly inspect if all is working well. \\ \\
	In 'render.py' there is only one function render() : this function load the mesh and call the polyscope function to add the cloud point and the mesh to scene before displaying it. PyMeshLab have a built-in function .show\_polyscope that show a mesh but working around with polyscope with find it more interesting to implement it with the polyscope function to have more control if with to update later. You see the result of the mesh visualisation script at Figure \ref{fig:ant-mesh}. We are happy of the result of the first step. The 3D mesh visualisation windows was one that we preferred among all of the other libraries tested. It have a lot of function and a button to screenshot that make it easier to make the report.


\begin{figure}
\begin{center}
  \includegraphics[width=0.5\textwidth]{ant}
  \caption{Visualisation of an ant mesh}
  \label{fig:ant-mesh}
  \end{center}
\end{figure}

%\section*{Acknowledgements}
%Anyone to thank/credit for helping your team along the way? This is the place to do it.

\medskip

\printbibliography

\end{document}
